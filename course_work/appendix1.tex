\section{Приложение А}
Реалицазия структуры \texttt{Queue} и некоторых методов для работы с ней:
\begin{verbatim}
//определение элемента Queue (Queue_Element):
typedef struct Queue_Element{
  int data;
  struct Queue_Element *next;
} Queue_Element;

//определение Queue:
typedef struct Queue{
  struct Queue_Element *head;
  struct Queue_Element *tail;
  int size;
} Queue;

//инициализация Queue:
Queue* Queue_init() {
  Queue *tmp = (Queue*) malloc(sizeof(Queue));
  tmp -> size = 0;
  tmp -> head = tmp -> tail = NULL;
  return tmp;
}

//дабавление элемента:
void Queue_push(Queue *q, int d) {
  Queue_Element *new_element = (Queue_Element*)
                            malloc(sizeof(Queue_Element));
  new_element -> data = d;
  if (q -> head == NULL){
    q -> head = q -> tail = new_element;
    q -> head -> next = q->tail->next= NULL;
  }else{
    q -> tail -> next = new_element;
    q -> tail = new_element;
  }
  q -> size++;
}

//удалаение Queue:
void Queue_free(Queue *q){
  Queue_Element* tmp; int i;
  for (i=0; i < q->size; i++){
    tmp = q -> head;
    (q -> head) = (q -> head -> next);
    free(tmp);
  }
  free(q);
}
\end{verbatim}
