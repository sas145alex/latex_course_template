\section{Введение}
Данная курсовая работа посвящена разработке интерфейсов информационной системы библиотечных залов,
позволяющий вести учет ее элементов и осуществлять надлежащий контроль связей между ними.
Также будет реализован поиск книг и осуществлена адаптивность для страниц веб-приложения.

Основным инструментом в разработке проекта является фреймворк Ruby on Rails,
написанный на языке программирования Ruby.
Данный фреймворк реализует архитектурный шаблон MVC (Model-View-Controller),
а также обеспечивает поддержку самых востребованных СУБД на сегодняшний день.
В качестве сервера базы данных нами будет использоваться PostgreSQL.

В качестве дополнительных гемов к фреймворку будем использовать гем haml для
более удобной шаблонизации веб-страниц и их отдельных частей. А так же гем
cocoon который поможет нам управлять вложенными формами.

Для подготовки пояснительной записки применялся набор компьютерной верстки \LaTeX.\

\subsection{Постановка задачи}

\begin{enumerate}
\item
Разработайть и добавить в базовое приложение модели необходимые из предметной области.
Описать требуемые ограничения целлостности.
\item
Организовать адаптивные интерфейсы редактирования. А так же настроить систему
авторизации и навигации для данных интерфейсов.
\item
Реализовать поиск книг по всем ее полям, а также всем полям доступных ей объектов.
\end{enumerate}
